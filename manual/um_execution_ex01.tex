%
% Example input file 1. Let say we want to calculate the optical depth scale of a MHD simulation snapshot...
% ... stored in the model atmosphere file: mhd_modeltau.bin, which has 512x512x492 grid points

These examples are provided in the {\it tests} directory together with the standard release.\\

In the first one, we just re-calculate the gas pressure and density for a model atmosphere:

\begin{ifbox}[label={tb:example01}]{{\it Example 1: Get gas pressure and density in HE}}
  \scriptsize
FILEMODEL:\\
test\_atmos.bin\\
MODE:\\
hydrostatic\\
HYDROSTATIC:\\
yes\\
MISC SETUP:\\
MODPATH ../../default/\\
SLDPATH ../../default/\\
END:
  \normalsize
\end{ifbox}

%{\Huge missing comments on the purpose. Maybe relate it to ipython notebooks?}

%
% Example input file 2. ...
% ... 

In this second example, we calculate the optical depth scale at 5000 {\AA} for the input model atmosphere used in the previous example, i.e. without the assumption of hydrostatic equilibrium.\\

\begin{ifbox}[label={tb:example02}]{{\it Example 2: Get optical depth}}
  \scriptsize
FILEMODEL:\\
test\_atmos.bin\\
MODE:\\
tau\\
HYDROSTATIC:\\
no\\
MISC SETUP:\\
MODPATH ../../default/\\
SLDPATH ../../default/\\
END:
  \normalsize
\end{ifbox}

%{\Huge missing comments on the purpose. Maybe relate it to ipython notebooks?}

%
% Example input file 3. ...
% ... 

Now, we combine the two previous cases calculating the optical depth scale of a model atmosphere whose gas pressure and density are previously calculated assuming hydrostatic equilibrium.

\begin{ifbox}[label={tb:example03}]{{\it Example 3: Get optical depth, HE}}
  \scriptsize
FILEMODEL:\\
test\_atmos.bin\\
MODE:\\
tau\\
HYDROSTATIC:\\
yes\\
MISC SETUP:\\
MODPATH ../../default/\\
SLDPATH ../../default/\\
END:
  \normalsize
\end{ifbox}

%{\Huge missing comments on the purpose. Maybe relate it to ipython notebooks?}

%
% Example input file 4. ...
% ... 

Now we consider the case of synthesising some Stokes profiles. To do so, we are going to synthesis the full Stokes vector for the blended lines 6301.5 and 6302.5 {\AA}.

\begin{ifbox}[label={tb:example04}]{{\it Example 4: Synthesis}}
  \scriptsize
LINES:\\
4,5 500 -700.0 5.0\\
FILEPROFILE:\\
syn\_profile.bin\\
FILEMODEL:\\
test\_atmos.bin\\
MODE:\\
synthesis\\
HYDROSTATIC:\\
no\\
STOKES SETUP:\\
STKI\\
STKQ\\
STKU\\
STKV\\
MISC SETUP:\\
MODPATH ../../default/\\
SLDPATH ../../default/\\
OUTPATH ./out\_dir/\\
END:
  \normalsize
\end{ifbox}

%{\Huge missing comments on the purpose. Maybe relate it to ipython notebooks?}

%
% Example input file 5. ...
% ... 

Now we proceed exactly on the same way, but assuming hydrostatic equilibrium.

\begin{ifbox}[label={tb:example05}]{{\it Example 5: Synthesis, HE}}
  \scriptsize
LINES:\\
4,5 500 -700.0 5.0\\
FILEPROFILE:\\
syn\_profile.bin\\
FILEMODEL:\\
test\_atmos.bin\\
MODE:\\
synthesis\\
HYDROSTATIC:\\
yes\\
MISC SETUP:\\
MODPATH ../../default/\\
SLDPATH ../../default/\\
OUTPATH ./out\_dir/\\
END:
  \normalsize
\end{ifbox}

Note here that we do not specify any Stokes parameter to be synthesis as by default all of them are calculated.\\

In the following example \ref{tb:example06} we consider the inversion of all the Stokes profiles for the temperature and the LOS velocity:\\

%
% Example input file 6. ...
% ... 

\begin{ifbox}[label={tb:example06}]{{\it Example 6: Inversion}}
  \scriptsize
LINES:\\
4,5 500 -700.0 5.0\\
FILEPROFILE:\\
syn\_profile.bin\\
FILEMODEL:\\
test\_atmos.bin\\
MODE:\\
inversion\\
HYDROSTATIC:\\
yes\\
STOKES SETUP:\\
STKI\\
STKQ\\
STKU\\
STKV\\
INVERSION SETUP:\\
TEM\\
VLOS\\
MISC SETUP:\\
DATAPATH ../test\_syn1/out\_dir/\\
MODPATH ../../default/\\
SLDPATH ../../default/\\
OUTPATH ./out\_dir/\\
END:
  \normalsize
\end{ifbox}

The result of the inversion is stores in two files. For the model, the output will be written in the {\it OUTPATH} directory with the same name as the input one preceded by the suffix: "out\_" (i.e. in this case, it would be written in "out\_dir\/" with the name: "out\_test\_atmos.bin"). Similarly, the synthetic profiles of the fitting, will be stored in the same directory ({\it OUTPATH}) and the name is the same as the input profiles with the additional "out\_" suffix (in this example: "out\_syn\_profile.bin").\\


%
% Example input file 7. ...
% ... 

In order to get the response functions, one has to set {\it MODE:} to {\it SYNTHESIS} and {\it WRITE RESPONSE FUNCTION:}
to {\it YES}. Additionally, we have to include {\it INVERSION SETUP:} in order to tell the code to what physical parameter we want the response functions to. In the example below, we would get the response functions to temperature ({\it TEM}), to LOS velocity ({\it VLOS}), and to the $z$ component of the magnetic field vector ({\it BZ}).//

\begin{ifbox}[label={tb:example07}]{{\it Example 7: Response functions}}
  \scriptsize
LINES:\\
4,5 500 -700.0 5.0\\
FILEPROFILE:\\
syn\_profile.bin\\
FILEMODEL:\\
test\_atmos.bin\\
MODE:\\
synthesis\\
WRITE RESPONSE FUNCTION:\\
YES\\
INVERSION SETUP:\\
TEM\\
VLOS\\
BZ\\
MISC SETUP:\\
DATAPATH ../test\_syn1/out\_dir/\\
MODPATH ../../default/\\
SLDPATH ../../default/\\
OUTPATH ./out\_dir/\\
END:
  \normalsize
\end{ifbox}




%
% Example input file N. ...
% ... 

%\begin{ifbox}[label={tb:example0N}]{{\it Example N}}
%  \scriptsize
%  \normalsize
%\end{ifbox}



