%
\label{sect:execution}
%
%%%% TMP:% General outline:
%%%% TMP:The execution of the FIRTEZ-dz code is performed as:\\
%%%% TMP:
%%%% TMP:\$ {\it mpirun -n \# ./firtez-dz.x control\_file.dat}\\
%%%% TMP:
%%%% TMP:Where {\it \#} must be substituted for whatever number of threads you want to run. Currently, FIRTEZ-dz has been tested in clusters with up to 1000 nodes without issues.\\
%%%% TMP:
%%%% TMP:For remote jobs with long execution times we recommend to run FIRTEZ-dz as:\\
%%%% TMP:
%%%% TMP:\$ {\it nohup mpirun -n \# ./firtez-dz.x control\_file.dat \&}\\
%%%% TMP:
%%%% TMP:This way the job will not stop should your internet connection drop.\\
%%%% TMP:
%%%% TMP:While FIRTEZ-dz is running, it also produces a lot of standard output on the console. Some of this output includes error messages and how to solve them. Therefore we do not recommend to run in silent mode ( {\it $>>$ /dev/null}). Then again, for remote jobs this might be your only option if the connection is intermitted. In this case, if you still want to run have a log file (i.e. {\it output.log}) to see what could have gone wrong, we recommend to run the job as:\\
%%%% TMP:
%%%% TMP:\$ {\it nohup mpirun -n \# ./firtez-dz.x control\_file.dat | tee output.log \&}\\
%%%% TMP:
%%%% TMP:When executing the FIRTE-DZ code, besides the executable itself ({\it firtez-dz.x}), we will need access to two additional files: the input/control file and the spectral line database file. The input/control file is commonly named {\it control\_file.dat} but you can choose to rename it as long as you provide the new name in the execution line above. The spectral line database file must always be named as {\it lines\_database.dat} as this is hardcoded inside FIRTEZ-dz. However, you can modify this file (see below) to add/remove spectral lines.
%%%% TMP:
%%%% TMP:
%%%% TMP:

{\it The code is able to perform different task for which, there are two mandatory files in addition to the executable itself: the input file and the spectral line database file.}
%
% Input file:
\subsection{Input file:}
It is a plain text file in which there are some mandatory fields and some optional ones. Each field is defined starting with its name (in capital letters) followed by colon and ends whenever another field starts or the end is reached, which is denoted with: {\it END:}. An example is provided after explaining each of the possible fields.\\

The following list covers ALL the fields the code works with, anything outside is omitted. The order with which the various {\it fields} appear does not matter.\\
\begin{itemize}
  \item {\it LINES:} {\bf Mandatory}\\
  It specifies the line/lines from the ones provided in the spectral lines database file (see Sec. \ref{sec:spectral_line_database}). It might contain as many spectral regions as desired and the format to supply them is as follows:\\
  index  nnn   init    ddd\\
  where the various parameters are:
  \begin{itemize}
    \item {\it index} refers to the spectral line index or indexes of the spectral line database to be synthesised together. If one is supplied the RTE is solved for that line. If several indexes (separated by comma) are supplied, then the RTE is solved for the whole set simultaneously.
    \item {\it nnn} (integer) specifies the number of spectral points to be solved.
    \item {\it init} (float) specifies the difference with respect to the reference wavelength set in the spectral line database (in m{\AA}) of the first wavelength point to consider. For the case of blended spectral regions, the reference wavelength is the one appearing in the spectral line database for the very first of the spectral lines specified.
    \item {\it ddd} (float) is the wavelength step in between each wavelength point in m{\AA}.
  \end{itemize}
  An example of such field would be as follows:\\
\begin{ifbox}[label={tb:lines}]{{\it Lines}}
  \scriptsize
  \ldots\\
  LINES:\\
  4 10 -123.45 6.7\\
  4,5 15 -234.56 7.8\\
  9 25 -3456.78 90.1\\
  END:\\
  \ldots
  \normalsize
\end{ifbox}
  In the first line, we specify that the spectral line identified with index {\it 4} in the spectral line database file is sampled in {\it 10} spectral points starting from {\it 123.45} m{\AA} of the line core (value specified in the spectral line database file) with a spectral sample of {\it 6.7} m{\AA}.
  In the second line, we also use the same spectral line as before ({\it 4} in the spectral line database file) but now, it is blended with another spectral line ({\it 5} in the spectral line database file). This spectral region is sampled at {\it 15} spectral points starting from {\it -234.56} m{\AA} of the line core of the first index ({\it 4}) (specified in the spectral line database file) with a spectral sample of {\it 7.8} m{\AA}.
  Finally, we consider a third spectral line ({\it 9} in the spectral line database file) which is sampled in 25 wavelength points starting from {\it -3456.78} of the core of line {\it 9} and equally spaced by {\it 90.1}.\\
  \item {\it BOX:} {\bf Optional}\\
  This field specifies the spatial dimension of the considered atmosphere to be handled. It has three mandatory lines that refer, respectively, to the {\it x}, {\it y}, and {\it z} spatial dimension and the three of them follow the same format:\\
  nn   dd\\
  where {\it nn} is the number of points (integer) in each dimension and {\it dd } is the step (real number, in km) sampling along each spatial direction. As an example, let us consider the following piece of input file:\\
\begin{ifbox}[label={tb:box}]{{\it Box}}
  \scriptsize
  \ldots\\
  BOX:\\
  2  9.01\\
  23  78.9\\
  512  1.234\\
  FILEMODEL:\\
  \ldots
  \normalsize
\end{ifbox}
  Here we would be considering a piece of atmosphere extended 2 spatial grid points in {\it x} dimension at a distance of {\it 9.01} km from each other, {\it 23 } spatial points in the {\it y} direction sampled every {\it 78.9} km, and finally, covering {\it 512} km in {\it z} direction every {\it 1.234} km. Notice here that in contrast to the previous piece of example, this field section ends as another field ({\it FILEMODEL:}) begins, i.e., not every field must end with {\it END:}.\\
  \item {\it FILEPROFILE:} {\bf Mandatory in {\it INVERSION} and {\it SYNTHESIS} mode, otherwise Inactive}\\
  It is a string containing the name of the file where spectral profiles are to be written. It is an optional field BUT it is mandatory for the synthesis and inversion modes. For example, if we would like to store the synthetic profiles for a supplied atmosphere in a file named {\it synthetic\_profile.bin}, we would write the following:\\
\begin{ifbox}[label={tb:box}]{{\it Box}}
  \scriptsize
  \ldots\\
  BOX:\\
  synthetic\_profile.bin\\
  FILEMODEL:\\
  \ldots
  \normalsize
\end{ifbox}
  \item {\it FILEMODEL:} {\bf Mandatory}\\
  This field is also a string containing the name of the model atmosphere to deal with. It is a MANDATORY field. For example, if our atmosphere is in {it model\_atmosphere.bin}, then we would have:\\
\begin{ifbox}[label={tb:box}]{{\it Box}}
  \scriptsize
  \ldots\\
  BOX:\\
  model\_atmosphere.bin\\
  MODE:\\
  \ldots
  \normalsize
\end{ifbox}
  \item {\it HYDROSTATIC:} {\bf Optional (default: {\it NO})}\\
  It is not a mandatory field and it can be {\it YES} or {\it NO}. If not specified, default is {\it NO}. If {\it YES}, the atmosphere is assumed in hydrostatic equilibrium, which means that, provided a boundary condition, it proceeds with the integration of the pressure using a Runge-Kutta 4 obtaining the gas pressure and density stratifications under hydrostatic equilibrium. If {\it NO}, then it takes the supplied gas pressures and densities to proceed with the resolution of the radiative transfer equation\footnote{Notice that either gas pressure or density are modified so that the equation of state is fulfilled at every height.}. For example, if we want to assume hydrostatic equilibrium, we would have the following:\\
\begin{ifbox}[label={tb:hydrostatic}]{{\it Hydrostatic}}
  \scriptsize
  \ldots\\
  HYDROSTATIC:\\
  YES\\
  MODE:\\
  \ldots
  \normalsize
\end{ifbox}
  \item {\it MODE:} {\bf Mandatory}\\
  This input tells the code what has to be done. There are currently 4 possibilities:
  \begin{itemize}
    \item {\it SYNTHESIS}\\
    It uses the model atmosphere supplied to solve the radiative transfer equation and get the synthetic Stokes spectra. In this mode, {\it FILEPROFILE:} is mandatory.
    \item {\it INVERSION}\\
    It uses the model atmosphere supplied and the profiles supplied to retrieve the model atmosphere that best fits the provided Stokes spectra. In this mode, {\it FILEPROFILE:} and {\it INVERSION SETUP:} are mandatory.
    \item {\it TAU}\\
    It uses the provided model atmosphere to estimate the $\lg\tau_{5}$ scale.
    \item {\it HYDROSTATIC}\\
    It uses the provided temperature and gas pressure boundary condition to solve hydrostatic equilibrium equation. It requires that {\it HYDROSTATIC:} field is {\it YES}.
  \end{itemize}
  \item {\it STOKES SETUP:} {\bf Optional (default: all active)}\\
  Optional field that specifies Stokes parameters to be considered. 
  It is useless except for {\it SYNTHESIS} and {\it INVERSION} mode. It specifies the Stokes parameters to be synthesis/inverted and/or the weights (only for {\it INVERSION}) with which one would like to perform the inversion. There might appear up to 4 lines inside this field with any/all of the following: {\it STKI}, {\it STKQ}, {\it STKU}, and/or {\it STKV}. By default all of them are synthesised/inverted and for the inversion, by default weights are set to 1. For example, if one wants to synthesis Stokes I and V for a given model atmosphere, then {\it STOKES SETUP:} field should appear as follows:\\
\begin{ifbox}[label={tb:stokes_setup1}]{{\it Stokes setup (1)}}
  \scriptsize
  \ldots\\
  STOKES SETUP:\\
  STKI\\
  STKV\\
  MODE:\\
  \ldots
  \normalsize
\end{ifbox}
  And if for instance, what we want is to invert all the Stokes parameters we might not include this field at all. If we want to set different weights for each Stokes parameter though, we must include this field as follows:\\
\begin{ifbox}[label={tb:stokes_setup2}]{{\it Stokes setup (2)}}
  \scriptsize
  \ldots\\
  STOKES SETUP:\\
  STKI\\
  STKV  2.4\\
  STKQ  4.2\\
  STKU  5.9\\
  MODE:\\
  \ldots
  \normalsize
\end{ifbox}
  where we would be setting the weight of Stokes I to 1 (default value), of Stokes Q to {\it 4.2}, {\it 5.9} for Stokes U, and {\it 2.4} for Stokes V. Notice here that weights appear quadratically in the definition of $\chi^2$:
  \begin{equation}
  \chi^2\propto\sum_{I,Q,U,V}\frac{\sum_{\lambda}(O_{\lambda}-I^{syn}_{\lambda})^2*w_{I,Q,U,V}^2}{\sigma_{I,Q,U,V}^2}
  \end{equation}
  \item {\it INVERSION SETUP:} {\bf Mandatory in {\it INVERSION} mode, otherwise Inactive}\\
  This field is mandatory for {\it INVERSION} mode. It can take any or almost all of the combinations between: {\it TEM}, {\it P0}, {\it PGAS}, {\it RHO}, {\it BX}, {\it BY}, {\it BZ}, and/or {\it VLOS}. It is NOT possible to set {\it TEM}, {\it PGAS}, and {\it RHO} at the same time since it might lead to combination of thermodynamical parameters that do not fulfil the equation of state. It is also not possible to invert at the same time {\it P0} with {\it PGAS} or {\it RHO}, since the former requires of assuming hydrostatic equilibrium. If {\it P0} is active with {\it PGAS}, {\it RHO}, or both, execution will stop. If {\it P0} is active and {\it HYDROSTATIC} is {\it NO} the code will stop execution. For the inversion, by default, the code will assume that one wants to invert all the available slabs. It is possible to limit the number of perturbations to be calculated by physical parameter by adding an accompanying number to each physical parameter.\\
%  {\it POW} 2\\
%  {\it MAXITER} 2\\
  For example, let say we want to invert the temperature, the three magnetic field components and line-of-sight velocity for a given profile. But, we want to let all the temperatures slabs as free parameters, only perturb each of the three magnetic field components at two points and the line-of-sight velocity at five points, then we should supply the following {\it INVERSION SETUP} field: \\
\begin{ifbox}[label={tb:inversion_setup}]{{\it Inversion setup}}
  \scriptsize
  \ldots\\
  INVERSION SETUP:\\
  TEM\\
  BX   2\\
  BY   2\\
  BZ   2\\
  VLOS  5\\
  END:\\
  \ldots
  \normalsize
\end{ifbox}
  Also it is possible to use {\it INVERSION SETUP:} field with {\it SYNTHESIS} mode when one wants to write down the response functions for a model atmosphere. For this only case, it is possible to set at the same time all of the following options: {\it TEM}, {\it PGAS}, {\it RHO}, {\it BX}, {\it BY}, {\it BZ}, and/or {\it VLOS}.\\
  \item {\it CONTINUUM NORMALIZATION:} {\bf Optional (default: active)}\\
  Optional field, useful for {\it SYNTHESIS} and {\it INVERSION} modes. Specifies whether continuum normalization to HSRA at disk center must be set or not. Default is {\it TRUE}. To disable it, one may use:\\
\begin{ifbox}[label={tb:continuum_normalization}]{{\it Continuum normalization}}
  \scriptsize
  \ldots\\
  CONTINUUM NORMALIZATION:\\
  FALSE\\
  END:\\
  \ldots
  \normalsize
\end{ifbox}
  \item {\it COUPLED INVERSION:} {\bf Optional (default: inactive)}\\
  Optional field only valid in {\it INVERSION} mode. This field is used for the inversion using Michiel van Noort's two-dimensional coupled inversion (van Noort, M. 2012). It requires at least one line, though some additional lines might be included in this field. The mandatory line specifies the file to contain the two-dimensional point-spread-function file name and should be written as follows:\\
\begin{ifbox}[label={tb:coupled_inversion1}]{{\it Coupled inversion (1)}}
  \scriptsize
  \ldots\\
  COUPLED INVERSION:\\
  FILENAME  psf2d.bin\\
  FILEPROFILE:\\
  \ldots
  \normalsize
\end{ifbox}
  In addition to that, there might be three additional input lines specifying the size of the block to consider ({\it BLCKSZ}), the number of threads to use for the matrix inversion ({\it NMTHRD}), or/and the radius (in pixels) of the point-spread-function to consider during the inversion {\it PSFRAD}.\\
\begin{ifbox}[label={tb:coupled_inversion1}]{{\it Coupled inversion (2)}}
  \scriptsize
  \ldots\\
  COUPLED INVERSION:\\
  FILENAME  psf2d.bin\\
  BLCKSZ 20\\
  NMTHRD 5\\
  PSFRAD 6\\
  FILEPROFILE:\\
  \ldots
  \normalsize
\end{ifbox}
  If any of the optional lines are missing, default values are taken: $BLCKSZ=10$, $NMTHRD=$ as many as MPI slaves, and $PSFRAD=3$. Be aware that since these values depend on the point-spread-function used, default values might not be optimal for your case.
  \item {\it WRITE RESPONSE FUNCTION:} {\bf Optional (default: inactive)}\\
  This is only useful for the {\it SYNTHESIS} mode. It goes in combination with {\it INVERSION SETUP:} and it can be either {\it NO} (default) or {\it YES}. This option MUST be used very carefully since it writes down files that can be extremely huge. For instance, if we are writing down the response functions for all the available physical parameters of a model atmosphere with 512 times 512 times 492 points at 100 spectral points, we will require more than 670 GB!. If one really wants to write down the response functions (provided that {\it MODE} is set to {\it SYNTHESIS}) then input file must contain:
\begin{ifbox}[label={tb:write_response_function}]{{\it Write response function}}
  \scriptsize
  \ldots\\
  WRITE RESPONSE FUNCTION:\\
  YES\\
  MODELNAME:\\
  \ldots
  \normalsize
\end{ifbox}
And in addition, one can specify in {\it INVERSION SETUP:} for what physical parameters response functions must be calculated. For an example on how to use this field have a look at \ref{tb:example07}
  \item {\it LINE SPREAD FUNCTION:} {\bf Optional (default: inactive)}\\
  This field is optional and has effect only in {\it SYNTHESIS} and {\it INVERSION} mode. It specifies the gaussian function to use to simulate the effect of the instrumental transmission function. It has to be specified for each of the spectral ranges considered. The format to provide this input parameter is:\\
  ss  ww,\\
  where {\it ss} is the standard deviation of the instrumental line-spread-function (in m{\AA}) and {\it ww} is the offset from the zero wavelength reference (in m{\AA}). Following the example shown in \bref{tb:lines}, we would like to set an instrumental line-spread-function of Gaussian shape with a standard deviation of {\it 3.5} for the first two spectral ranges (starting spectral line index {\it 4}) and {\it 7.6} for the last one (spectral line index {\it 9}), and additionally, the offset for the middle spectral region is going to be 1.4 m{\AA}:\\
\begin{ifbox}[label={tb:line_spread_function}]{{\it Line spread function}}
  \scriptsize
  \ldots\\
  LINE SPREAD FUNCTION:\\
  3.5 0.\\
  3.5 1.4\\
  7.6 0\\
  CONTINUUM NORMALIZATION:\\
  \ldots
  \normalsize
\end{ifbox}
  It is important to note that there MUST be the exactly same number of inputs as for the {\it LINES} field, otherwise, the code will stop.
  \item {\it MISC SETUP:} {\bf Optional (default: see below)}\\
  In addition to the previous inputs, there are some additional possibilities:
  \begin{enumerate}
    \item {\it SVDTOL} {\bf Optional (default: $10^{-4}$)}\\
    Used in {\it INVERSION} mode. It defines the threshold used for the number of eigen-values used during the inversion of the dampened Hessian matrix.
    \item {\it NOISE} {\bf Optional (default: $10^{-4}$)}\\
    Used in {\it INVERSION} mode. It defines the standard deviation/noise of the data to be used during the inversion and error calculation.
    \item {\it TEMP} {\bf Optional (default: Inactive)}\\
    Used in {\it INVERSION} mode. It pre-calculates an initial temperature perturbation based on the average intensity of the first spectral region provided.
    \item {\it PGAS} {\bf Optional (default: Inactive)}\\
    Used in {\it INVERSION} mode. It pre-calculates an initial gas pressure perturbation so that $\lg\tau_{5}$ is formed inside the provided height range.
    \item {\it BVEC} {\bf Optional (default: Inactive)}\\
    Used in {\it INVERSION} mode. It pre-calculates an initial magnetic field vector perturbation based on the four Stokes spectra of the first spectral region provided (Only works if inverting the four Stokes parameters, otherwise ignored).
    \item {\it VLOS} {\bf Optional (default: Inactive)}\\
    Used in {\it INVERSION} mode. It pre-calculates an initial line-of-sight velocity component perturbation based on the Stokes I spectra of the first spectral region provided (Only works if Stokes I is inverted, otherwise ignored).
    \item {\it FULL\_STOKES} {\bf Optional (default: Inactive)}\\
    Used in {\it INVERSION} and {\it SYNTHESIS} mode. It stores (and writes down) the Stokes parameters for the whole three-dimensional volume considered. This option may considerably increase the memory requirement.
    \item {\it SLDPATH} {\bf Optional (default: ./)}\\
    It specifies the directory where the file with the spectral line data is.
    \item {\it MODPATH} {\bf Optional (default: ./)}\\
    It specifies the directory where the file with the model atmosphere is.
    \item {\it DATAPATH} {\bf Optional (default: ./)}\\
    It specifies the directory where the file with the stokes spectra is (only useful in inversion mode).
    \item {\it OUTPATH} {\bf Optional (default: ./)}\\
    It specifies the directory where the output data will be written, all of it, i.e. the model atmosphere, the synthetic profiles, the response functions (if required).
  \end{enumerate}
  \item {\it END:}\\
  It specifies the end of the input file that must be read by the code. Anything below is ignored.
\end{itemize}
%
% Some input file examples:
\subsection{Input file examples:}
%
% Example input file 1. Let say we want to calculate the optical depth scale of a MHD simulation snapshot...
% ... stored in the model atmosphere file: mhd_modeltau.bin, which has 512x512x492 grid points

These examples are provided in the {\it tests} directory together with the standard release.\\

In the first one, we just re-calculate the gas pressure and density for a model atmosphere:

\begin{ifbox}[label={tb:example01}]{{\it Example 1}}
  \scriptsize
FILEMODEL:\\
test\_atmos.bin\\
MODE:\\
hydrostatic\\
HYDROSTATIC:\\
yes\\
MISC SETUP:\\
MODPATH ../../default/\\
SLDPATH ../../default/\\
END:
  \normalsize
\end{ifbox}

%{\Huge missing comments on the purpose. Maybe relate it to ipython notebooks?}

%
% Example input file 2. ...
% ... 

In this second example, we calculate the optical depth scale at 5000 {\AA} for the input model atmosphere used in the previous example, i.e. without the assumption of hydrostatic equilibrium.\\

\begin{ifbox}[label={tb:example02}]{{\it Example 2}}
  \scriptsize
FILEMODEL:\\
test\_atmos.bin\\
MODE:\\
tau\\
HYDROSTATIC:\\
no\\
MISC SETUP:\\
MODPATH ../../default/\\
SLDPATH ../../default/\\
END:
  \normalsize
\end{ifbox}

%{\Huge missing comments on the purpose. Maybe relate it to ipython notebooks?}

%
% Example input file 3. ...
% ... 

Now, we combine the two previous cases calculating the optical depth scale of a model atmosphere whose gas pressure and density are previously calculated assuming hydrostatic equilibrium.

\begin{ifbox}[label={tb:example03}]{{\it Example 3}}
  \scriptsize
FILEMODEL:\\
test\_atmos.bin\\
MODE:\\
tau\\
HYDROSTATIC:\\
yes\\
MISC SETUP:\\
MODPATH ../../default/\\
SLDPATH ../../default/\\
END:
  \normalsize
\end{ifbox}

%{\Huge missing comments on the purpose. Maybe relate it to ipython notebooks?}

%
% Example input file 4. ...
% ... 

Now we consider the case of synthesising some Stokes profiles. To do so, we are going to synthesis the full Stokes vector for the blended lines 6301.5 and 6302.5 {\AA}.

\begin{ifbox}[label={tb:example04}]{{\it Example 4}}
  \scriptsize
LINES:\\
4,5 500 -700.0 5.0\\
FILEPROFILE:\\
syn\_profile.bin\\
FILEMODEL:\\
test\_atmos.bin\\
MODE:\\
synthesis\\
HYDROSTATIC:\\
no\\
STOKES SETUP:\\
STKI\\
STKQ\\
STKU\\
STKV\\
MISC SETUP:\\
MODPATH ../../default/\\
SLDPATH ../../default/\\
OUTPATH ./out\_dir/\\
END:
  \normalsize
\end{ifbox}

%{\Huge missing comments on the purpose. Maybe relate it to ipython notebooks?}

%
% Example input file 5. ...
% ... 

Now we proceed exactly on the same way, but assuming hydrostatic equilibrium.

\begin{ifbox}[label={tb:example05}]{{\it Example 5}}
  \scriptsize
LINES:\\
4,5 500 -700.0 5.0\\
FILEPROFILE:\\
syn\_profile.bin\\
FILEMODEL:\\
test\_atmos.bin\\
MODE:\\
synthesis\\
HYDROSTATIC:\\
yes\\
MISC SETUP:\\
MODPATH ../../default/\\
SLDPATH ../../default/\\
OUTPATH ./out\_dir/\\
END:
  \normalsize
\end{ifbox}

Note here that we do not specify any Stokes parameter to be synthesis as by default all of them are calculated.

%
% Example input file 6. ...
% ... 

\begin{ifbox}[label={tb:example06}]{{\it Example 6}}
  \scriptsize
LINES:\\
4,5 500 -700.0 5.0\\
FILEPROFILE:\\
syn\_profile.bin\\
FILEMODEL:\\
test\_atmos.bin\\
MODE:\\
inversion\\
HYDROSTATIC:\\
yes\\
STOKES SETUP:\\
STKI\\
STKQ\\
STKU\\
STKV\\
INVERSION SETUP:\\
TEM\\
VLOS\\
MISC SETUP:\\
DATAPATH ../test\_syn1/out\_dir/\\
MODPATH ../../default/\\
SLDPATH ../../default/\\
OUTPATH ./out\_dir/\\
END:
  \normalsize
\end{ifbox}




%
% Example input file 7. ...
% ... 

\begin{ifbox}[label={tb:example07}]{{\it Example 7}}
  \scriptsize
  \normalsize
\end{ifbox}




%
% Example input file N. ...
% ... 

\begin{ifbox}[label={tb:example0N}]{{\it Example N}}
  \scriptsize
  \normalsize
\end{ifbox}




%
Now that we have covered the input file, let us have a look at the spectral lines database file.\\
% Spectral lines database file:
\subsection{Spectral lines database file:}

% Spectral lines database file:
\label{sec:spectral_line_database}
%

Spectral line database is the file in which spectral lines atomic data is supplied to the code. It is very similar to SIR code LINES file, with some differences though. Each row specifies the data for a given atomic transition and the various columns are:
\begin{enumerate}
  \item Index: An integer number identifying each atomic data entry. It can be any integer.
  \item Atomic number (Z).
  \item Ionisation stage: Only lines in neutral, once, and twice ionization stages are allowed. Adimensional.
  \item Central wavelength of the transition: float number, in angstrom.
  \item Lower energy level: the format is as follows: X.XCY.Y, where X.X refers to the spin momentum, C refers to the angular momentum, and Y.Y to the $J$ quantum number of the lower level.
  \item Upper energy level: the format is as follows: X.XCY.Y, where X.X refers to the spin momentum, C refers to the angular momentum, and Y.Y to the $J$ quantum number of the upper level.
  \item Oscillator strength ($\log gf$): real number (might be positive or negative)
  \item Ionization energy (in eV) of the lower level: real number
  \item Barklem collisional theory velocity parameter ($\alpha$): real number, adimensional.
  \item Barklem collisional theory cross-section: real number in atomic units.
  \item Barklem collisional theory transition type: two character string.
  \item Barklem collisional theory lower level energy limit: real number in ${\rm cm}^{-1}$.
  \item Barklem collisional theory upper level energy limit: real number in ${\rm cm}^{-1}$.
\end{enumerate}

Some comments on these last five parameters (the ones related to ABO collisional theory) are required. If the user provides the transition type and the other are set to 0.0, then the tables published by Barklem and co-authors are interpolated to the effective principal quantum number ($n^{*}$). If the user supplies the velocity parameter ($\alpha$) and the cross-section and the last two fields are 0.0, then, independently of the transition type supplied (yet two string characters must be supplied), the values provided for the velocity parameter ($\alpha$) and the cross-section are used. Finally, if the last two (lower and upper energy limits) are non-zero, transition type parameter is MANDATORY. In this case, the last two parameters (lower and upper energy limits) are used to calculate the effective principal quantum number ($n^{*}$) and then, using the transition type parameter, the ABO tables are interpolated accordingly. The values for the velocity parameter ($\alpha$) and cross-section for each entry of the spectral line database is stored in the output log file {\it info.log}.\\

Let use Fe{\sc I} 6173.3 {\AA} spectral line as an example.

The first possibility might be to provide the transition type, i.e. {\it sp} as:\\

\begin{sldbox}[label={tb:sldex01}]{{\it Spectral line database ex. 1}}
  \scriptsize
  72  026  1   6173.3354  5.0P1.0  5.0D0.0  -2.880  2.223  0.000  0000.0 sp 0.0  0.0
  \normalsize
\end{sldbox}

If one does so, one would retrieve a velocity parameter ($\alpha$) of 0.2406 and a cross-section value of 327.2273 and the values are wrong. It is so because for the energy levels involved in this transition the energy limit and the ionization energy are not the same. In this case, there are two possibilities: 1- We know the values for the velocity parameter ($\alpha$) and the cross-section:

\begin{sldbox}[label={tb:sldex02}]{{\it Spectral line database ex. 2}}
  \scriptsize
  75  026  1   6173.3354  5.0P1.0  5.0D0.0  -2.880  2.223  0.2660  280.5707 aa 0.0  0.0
  \normalsize
\end{sldbox}

In this case, the three last parameters are ignored by the code. Or, 2- we calculate the energy limit of each level involved in the transition (in ${\rm cm}^{-1}$):

\begin{sldbox}[label={tb:sldex03}]{{\it Spectral line database ex. 3}}
  \scriptsize
  73  026  1   6173.3354  5.0P1.0  5.0D0.0  -2.880  2.223  0.000  0000.0 sp 77212.151  65610.304
  \normalsize
\end{sldbox}

As you can check at the beginning of {\it info.log} the ``Collisional cross section'' and ``Velocity parameter'' are the same as specified in \bref{tb:sldex02}. These last two cases are correct, yet the first one leads to incorrect results.



%

